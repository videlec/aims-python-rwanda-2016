\documentclass[a4paper]{article}

\newcounter{exo}

\newenvironment{exercise}%
{\par\vspace{.5\baselineskip}\noindent
\refstepcounter{exo}%
\textbf{Exercise \theexo}%
\par\vspace{.5\baselineskip}\noindent\ignorespaces
\begin{center}\begin{minipage}{0.9\linewidth}}%
{\end{minipage}\end{center}\smallskip}

\usepackage{amsmath}
\usepackage{paralist}
\usepackage[margin=2.5cm]{geometry}

\begin{document}
\begin{center}
{\huge Exercises week 1: arithmetic functions}
\end{center}

\bigskip

For this first sheet of exercises you are asked to create a Jupyter notebook in which you will program several arithmetic functions.

\smallskip

\begin{exercise}
Write a function \texttt{prod(l)} that returns the product of the elements in the list \texttt{l}. In case the list is empty, the function should return \texttt{1}.

\smallskip\noindent
For example
\begin{verbatim}
>>> prod([1, 3, 2, 3])
18
>>> prod([])
1
\end{verbatim}
\end{exercise}


\begin{exercise}
Write a function \texttt{gcd(x, y)} that computes the greatest common divisor of two integers using Euclide algorithm.

\smallskip\noindent
Plot the graphic of the function $n \mapsto \# \{(p,q): 1 \leq p \leq n,\ 1 \leq q \leq n,\ \gcd(p, q) = 1\}$.

\smallskip\noindent
Could you guess the asymptotic?
\end{exercise}

\bigskip

Recall that the \emph{digits} of an integer $n$ in base $b$ is the sequence of numbers $(n_0, n_1, \ldots, n_d)$
in $\{0, 1, \ldots, b-1\}$ so that
\[
n = \sum_{i=0}^d n_i b^i.
\]
For example, the digits of $12$ in base 10 are $(2,1)$ and in base $2$ are $(0,1,1)$.
\begin{exercise}
Write a function \texttt{digits(n, b)} that return the list of digits of the number \texttt{n} in base \texttt{b}.

\smallskip\noindent
Let $n = 123576537645123412$ written in base 10. What are its digits in base 2? In base 3?

\smallskip\noindent
What is the sum of the digits of $2^{100}$ written in base 3?
\end{exercise}

\begin{exercise}
Write a function \texttt{prime\_range(n)} that returns the list of prime numbers less than \texttt{n}. (\textit{hint: use the sieve of Eratosthenes})

\smallskip\noindent
Use your function \texttt{prime\_range} to answer the following questions
\begin{compactitem}
\item How many primes are less than 1243? less than 254321?
\item How many primes are there between 43123 and 122505?
\item Make a graphic of the function $n \mapsto \# \{p: p \leq n \ \text{and $p$ is prime}\}$.
Could you guess the asymptotic of this function?
\end{compactitem}
\end{exercise}

\newpage

We recall that the set of integers is a unique factorization domain (UFD): any non-zero integer can be uniquely written
 as a product $n = s\ p_0^{k_0}\ p_1^{k_1}\ \ldots\ p_m^{k_m}$, $s$ is the sign (ie either $+1$ or $-1$), $p_i$ are primes and $k_i$ are positive integers.
\begin{exercise}
Write a function \texttt{factor(n)} that returns the factorisation of a positive number \texttt{n} as a list of pairs
\texttt{[[p0, k0], [p1, k1], ..., [pm, km]]} where the \texttt{pi} and \texttt{ki} are respectively
the prime numbers and exponents.

\smallskip\noindent
For example
\begin{verbatim}
>>> factor(12)
[[2, 2], [3, 1]]
>>> factor(19)
[[19, 1]]
\end{verbatim}

\noindent
What are the factorisation of
\begin{compactitem}
\item 533850245821893?
\item $2^{97} + 1$?
\item $2^{97} - 1$?
\end{compactitem}
\end{exercise}

\begin{exercise}
A positive number $n$ is $k$-th power free if there is no integer $m$ greater than 1 so that $m^k$ divides $n$.
\begin{compactitem}
\item Write a function \texttt{is\_kth\_power\_free(n)} that tests whether the number $n$ is $k$-th power free.
\item Make the list of squarefree (ie $2$-nd power free) numbers less than 100.
\item Program the function \texttt{moebius\_mu(n)} that is the M\"obius $\mu$ function
$$n \mapsto
\left\{
\begin{array}{ll}
0 & \text{if $n$ is not squarefree} \\
(-1)^k & \text{if $n$ is a product of $k$ distinct prime numbers} \\
\end{array} \right.$$
\item Find the smallest integer $n \geq 2$ so that $\mu(n) = 1$, $\mu(n+1) = -1$ and $\mu(n) = 1$.
\end{compactitem}
\end{exercise}

\begin{exercise}
Write a function \texttt{sigma(n, k)} that return the sum of the $k$-th power of the divisors of $n$. For example
$\sigma(n, 0)$ is the number of divisors of $n$ and $\sigma(n, 1)$ is the sum of divisors. (\textit{hint: you should use the function \texttt{factor}})

\smallskip\noindent
Could you write a more efficient function \texttt{num\_divisors(n)} that is equivalent to \texttt{sigma(n, 0)}?

\smallskip\noindent
What is the number less than $1000$ that has the largest number of divisors?

\smallskip\noindent
What is the number less than $1000$ that has the largest sum of divisors?
\end{exercise}

\end{document}
