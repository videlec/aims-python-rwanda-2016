\documentclass[a4paper]{article}

\newcounter{exo}

\newenvironment{exercise}%
{\par\vspace{\baselineskip}\noindent
\refstepcounter{exo}%
\textbf{Exercise \theexo}%
\par\vspace{\baselineskip}\noindent\ignorespaces}%
{\smallskip}

\usepackage{paralist}
\usepackage[margin=2.5cm]{geometry}

\begin{document}
\thispagestyle{empty}
\begin{center}
{\huge Exercises week 1: arithmetic functions}
\end{center}

\bigskip

For this first sheet of exercises you are asked to create a Jupyter notebook in which you will program several arithmetic functions.

\smallskip

\begin{exercise}
Write a function \texttt{prod(l)} that returns the product of the elements in the list \texttt{l}. In case the list is empty, the function should return \texttt{1}.

\smallskip\noindent
For example
\begin{verbatim}
>>> prod([1, 3, 2, 3])
18
>>> prod([])
1
\end{verbatim}
\end{exercise}

\begin{exercise}
Write a function \texttt{digits(n, b)} that return the list of digits of the number \texttt{n} in base \texttt{b}.

\smallskip\noindent
Let $n = 123576537645123412$ written in base 10. What are its digits in base 2? In base 3?
\end{exercise}

\begin{exercise}
Write a function \texttt{prime\_range(n)} that return the list of prime numbers less than \texttt{n}. (\textit{hint: use the sieve of Eratosthenes})

\smallskip\noindent
Use the function \texttt{prime\_range} to answer the following questions
\begin{compactitem}
\item How many primes are less than 1243? less than 254321?
\item How many primes are there between 43123 and 122505?
\end{compactitem}
\end{exercise}

\begin{exercise}
Write a function \texttt{factor(n)} that returns the factorisation of \texttt{n} as a list of pairs \texttt{[[p0, k0], [p1, k1], ..., [pm, km]]} where the \texttt{pi} and \texttt{ki} are respectively the prime numbers and exponents.

\smallskip\noindent
For example
\begin{verbatim}
>>> factor(12)
[[2, 2], [3, 1]]
>>> factor(19)
[[19, 1]]
\end{verbatim}

\noindent
What are the factorisation of
\begin{compactitem}
\item 533850245821893?
\item $2^{97} + 1$?
\item $2^{97} - 1$?
\end{compactitem}
\end{exercise}

\begin{exercise}
Write a function \texttt{sigma(n, k)} that return the sum of the $k$-th power of the divisors of $n$. For example $\sigma(n, 0)$ is the number of divisors of $n$ and $\sigma(n, 1)$ is the sum of divisors. (\textit{hint: you should use the function \texttt{factor}})

\smallskip\noindent
Could you write a more efficient function \texttt{num\_divisors(n)} that is equivalent to \texttt{sigma(n, 0)}?

\smallskip\noindent
What is the number less than $1000$ that has the largest number of divisors?

\smallskip\noindent
What is the number less than $1000$ that has the largest sum of divisors?
\end{exercise}

\end{document}
