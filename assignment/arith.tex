\documentclass[a4paper]{article}

\newcounter{exo}

\newenvironment{exercise}%
{\par\vspace{.5\baselineskip}\noindent
\refstepcounter{exo}%
\textbf{Exercise \theexo}%
\par\vspace{.5\baselineskip}\noindent\ignorespaces
\begin{center}\begin{minipage}{0.9\linewidth}}%
{\end{minipage}\end{center}\smallskip}

\usepackage{amsmath}
\usepackage{paralist}
\usepackage[margin=2.5cm]{geometry}

\begin{document}
\begin{center}
\Large Scientific Software Development in Python, AIMS Rwanda 2016 \\ \smallskip
Assignment 1: arithmetic functions
\end{center}

\bigskip

For this first set of exercises you are asked to create a Jupyter notebook in
which you will program several arithmetic functions. There are also questions
for which the answer should appear in the notebook. Normally you have also
downloaded a notebook template to start from.

You are free to do the exercises in any order you like. However, each exercise
should start with a Markdown cell with the number of the exercise.

Each function you program should come with
\begin{itemize}
\item some examples
\item some tests that check that the result is consistent. For example if you have
programed the \texttt{gcd} function you can have the following in a cell
\begin{verbatim}
for a in range(50):
    for b in range(50)
        assert gcd(a, b) == gcd(b, a)
        assert gcd(a + b, b) == gcd(a, b)
        for c in range(10):
            assert gcd(c*a, c*b) == c * gcd(a, b)
\end{verbatim}
\end{itemize}

A lot of attention will be paid to the clarity of the code. In particular
\begin{compactitem}
\item give meaningful name to your variables, for example \texttt{s} for a
sum and \texttt{p} for a product, \texttt{counter} for a counter in
a \texttt{range}, etc
\item use comments (using \texttt{\#}) to explain the delicate steps
of your algorithms.
\end{compactitem}

\smallskip

One can also consider the digits of fractions. For example, in base 10 one can write
$$
\frac{1}{2} = 0.5
\qquad
\frac{1}{3} = 0.(3)
\qquad
\frac{27}{28} = 0.96(428571)
$$
where the part in paranthesis should be repeated periodically.
\begin{exercise}
Write a function \texttt{fraction\_digits(p, q, b)} that assumes that $p < q$
and returns a pair of lists \texttt{[l1, l2]} where:
\begin{compactitem}
\item \texttt{l1} is the list of digits that constitutes the preperiodic part of the expansion of $\frac{p}{q}$,
\item \texttt{l2} is the list of digits that constitutes the periodic part of the digits of $\frac{p}{q}$.
\end{compactitem}
For example
\begin{verbatim}
>>> fraction_digits(1, 2, 10)
[[5], []]
>>> fraction_digits(1, 3, 10)
[[], [3]]
>>> fraction_digits(27, 28, 10)
[[9, 6], [4, 2, 8, 5, 7, 1]]
\end{verbatim}
\end{exercise}

\begin{exercise}
Write a function \texttt{prime\_range(n)} that returns the list of prime
numbers less than \texttt{n}. (\textit{hint: use the sieve of Eratosthenes})

\smallskip\noindent
Use your function \texttt{prime\_range} to answer the following questions
\begin{compactitem}
\item How many primes are less than 1243? less than 254321?
\item How many primes are there between 43123 and 122505?
\item Make a graphic of the function $n \mapsto \# \{p: p \leq n \ \text{and $p$ is prime}\}$.
Could you guess the asymptotic of this function?
\end{compactitem}
\end{exercise}

We recall that the set of integers is a unique factorization domain (UFD): any
non-zero integer can be uniquely written as a product $n = s\ p_0^{k_0}\
p_1^{k_1}\ \ldots\ p_m^{k_m}$, $s$ is the sign (ie either $+1$ or $-1$), $p_i$
are primes and $k_i$ are positive integers.
\begin{exercise}
Write a function \texttt{factor(n)} that returns the factorisation of a
positive number \texttt{n} as a list of pairs \texttt{[[p0, k0], [p1, k1], \ldots,
[pm, km]]} where the \texttt{pi} and \texttt{ki} are respectively the prime
numbers and exponents.

\smallskip\noindent
For example
\begin{verbatim}
>>> factor(12)
[[2, 2], [3, 1]]
>>> factor(19)
[[19, 1]]
\end{verbatim}

\noindent
What are the factorisation of
\begin{compactitem}
\item 533850245821893?
\item $2^{97} + 1$?
\item $2^{97} - 1$?
\end{compactitem}
\end{exercise}

\begin{exercise}
A positive number $n$ is $k$-th power free if there is no integer $m$ greater
than 1 so that $m^k$ divides $n$.
\begin{compactitem}
\item Write a function \texttt{is\_kth\_power\_free(n)} that tests whether the number $n$ is $k$-th power free.
\item Make the list of squarefree (ie $2$-nd power free) numbers less than 100.
\item Program the function \texttt{moebius\_mu(n)} that is the M\"obius $\mu$ function
$$n \mapsto
\left\{
\begin{array}{ll}
0 & \text{if $n$ is not squarefree} \\
(-1)^k & \text{if $n$ is a product of $k$ distinct prime numbers} \\
\end{array} \right.$$
\item Find the smallest integer $n \geq 2$ so that $\mu(n) = 1$, $\mu(n+1) = -1$ and $\mu(n+2) = 1$.
\end{compactitem}
\end{exercise}

\begin{exercise}
Write a function \texttt{sigma(n, k)} that return the sum of the $k$-th power
of the divisors of $n$. For example $\sigma(n, 0)$ is the number of divisors of
$n$ and $\sigma(n, 1)$ is the sum of divisors. (\textit{hint: you should use
the function \texttt{factor}})

\smallskip\noindent
Could you write a more efficient function \texttt{num\_divisors(n)} that is equivalent to \texttt{sigma(n, 0)}?

\smallskip\noindent
What is the number less than $1000$ that has the largest number of divisors?

\smallskip\noindent
What is the number less than $1000$ that has the largest sum of divisors?
\end{exercise}

\end{document}
